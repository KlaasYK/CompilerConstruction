\documentclass[a4paper,11pt]{article}

\usepackage[utf8]{inputenc}
\usepackage[UKenglish]{babel}
\usepackage{fontenc}
\usepackage{graphicx}
\usepackage{amsmath}

\usepackage{listings}
\usepackage{color}
\usepackage{url}

\definecolor{mygreen}{rgb}{0,0.6,0}
\definecolor{lightgray}{rgb}{0.9,0.9,0.9}
\definecolor{mymauve}{rgb}{0.58,0,0.82}

\title{Compliler Construction: Infinite precision signed integers }
\author{Klaas Kliffen, Rik Schaaf}
\date{}

\begin{document}
\lstset{ %TODO: change this
language=C,
backgroundcolor=\color{lightgray},
basicstyle=\footnotesize,
breakatwhitespace=false,
breaklines=true,
commentstyle=\color{mygreen},
keywordstyle=\color{blue},
stringstyle=\color{muave},
tabsize=2,
}

\maketitle


\section{Problem description}
The aim of this lab is to give a library to use with a compiler for a language with infinite precision integers.
Furthermore the following operations on these integers must be implemented:
\begin{itemize}
 \item Creating an integer from string
 \item Printing an integer to stdout
 \item Adding two integers
 \item Subtracting two integers
 \item Multiplying two integers
 \item Dividing an integer by another integer
 \item Apply a modulo on an integer
 \item Apply a power to an integer
\end{itemize}

\section{Problem analysis}
First there must be a data structure to define the Big integers.
A logical solution is to store each digit of the number in an array as integers in C are not infinite.
To keep track of the sign, another numeric variable can be used.
For example a -1 for negative integers and 1 for positive.
When using an array representation, a length too can be useful, as C does provide a length function for integers.
\subsection{Creating from string}

\subsection{Printing an integer}

\subsection{Adding}

\subsection{Subtracting}

\subsection{Multiplying}
There are multiple algorithms to choose from when multiplying.
The standard long multiplication algorithm which runs in $O(n^2)$.
As there can be a lot of multiplications in a program, a lower order function is preferred.
An example discussed here is the karatsuba algorithm which uses a divide and conquer method.
In pseudo code it looks like this:
\begin{lstlisting}
procedure karatsuba(num1, num2)
  if (num1 < 10) or (num2 < 10)
    return num1*num2
  /* calculates the size of the numbers */
  m = max(size_base10(num1), size_base10(num2))
  m2 = m/2
  /* split the digit sequences about the middle */
  high1, low1 = split_at(num1, m2)
  high2, low2 = split_at(num2, m2)
  /* 3 calls made to numbers approximately half the size */
  z0 = karatsuba(low1,low2)
  z1 = karatsuba((low1+high1),(low2+high2))
  z2 = karatsuba(high1,high2)
  return (z2*10^(2*m2))+((z1-z2-z0)*10^(m2))+(z0)
\end{lstlisting}
Using this algorithm you perform three multiplications of integers with approximately half of the number of digits of large integers.
This should yield a complexity (for sufficiently high n) of $\Theta (n^{ \text{log} _2(3)})$.

\subsection{Division}
There are not many efficient division algorithms so the algorithm chosen here is the long division method.
%TODO: moeten we dit echt gaan uitleggen?
\subsection{Modulo}
The module operation is the counterpart of the devision method, so the same method for the long division is used here.

\subsection{Power function}
The straightforward method of exponentiation of multiplying k times for $n^k$ can be easily optimized.
For the power operation the exponentiation by squaring is used.
This vastly reduces the number of multiplying operations and runs in $O((n \text{log} (x))^{ \text{log} _2(3)}$ when using the karatsuba algorithm for multiplication.

%actual code discussion
\section{Design of solution}
The chosen data structure has been implemented as follows:
\begin{lstlisting}
struct EGCLint {
  uint32_t *digits; /* the value for each of the digits, big endian */
  unsigned long length; /* maximum value is approx. 10^4294967295, not infinite, but close enough. */
  int sign; /* wheter it is positive, of negative */
};
typedef struct EGCLint Integer;
\end{lstlisting}
\\
In the following algorithms, we try to copy every input integer when needed, preventing overwriting the values of the second operand.
\subsection{Creating from string}

\subsection{Printing an integer}

\subsection{Adding}

\subsection{Subtracting}

\subsection{Multiplying}
The karatsuba algorithm is implemented in a recursive manner:
\begin{lstlisting}
/* recursive karatsuba */
Integer karatsuba(Integer a, Integer b) {
  unsigned long alen = a.length, blen = b.length, m, m2;
  Integer z0, z1, z2, returnInt;
  
  /* base state of recursion */
  /* a < 10 */
  if (alen < 2) {
    deepCopyInteger(b, &returnInt);
    simpleMul(&returnInt, a);
    return returnInt;
  }
  if (blen < 2) {
    deepCopyInteger(a, &returnInt);
    simpleMul(&returnInt, b);
    return returnInt;
  }
  
  m = MAX(alen, blen);
  m2 = m / 2;
  
  Integer high1, low1, high2, low2;
  splitAt(&high1, &low1, a, m2);
  splitAt(&high2, &low2, b, m2);
  
  /* divide */
  z0 = karatsuba(low1, low2);
  z2 = karatsuba(high1, high2);
  addInteger(&low1, high1);
  addInteger(&low2, high2);
  /* the low Integers now contain the sums */
  z1 = karatsuba(low1, low2);
  
  /* now follows the z2 * 10^(2*m2) + ((z1-z2-z0)*10^(m2) + z0 */
  /* the z1 - z2 -z0 stored in the z1 */
  subInteger(&z1, z2);
  subInteger(&z1, z0);
  
  /* shift right for z1 and z2 */
  shiftRight(&z1, m2);
  shiftRight(&z2, m2 * 2);
  
  /* and conquer */
  addInteger(&z2, z1);
  addInteger(&z2, z0);

  /* free the temp integers */
  freeInteger(&high1);
  freeInteger(&low1);
  freeInteger(&high2);
  freeInteger(&low2);
  /* done */
  deepCopyInteger(z2, &returnInt);
  freeInteger(&z0);
  freeInteger(&z1);
  freeInteger(&z2);
  
  return returnInt;
}
\end{lstlisting}
The base state of the recursion occurs when one of the operand is a single digit number.
In this state it is possible to multiply the other operand in a simple loop, adding a carry if needed.\\
In the recursion both integers are split and the low and high order bits are stored.
After that z0, z1 and z2 are calculated.
Before the adding, z2 and z1 are shifted tot the right (shift left in little endian).
This is the multiplication with a power of the base.
After this the results can be added together to get the desired value.\\
In the main function the 
\subsection{Division}

\subsection{Modulo}

\subsection{Power function}

\section{Evaluation}
The performance of the algorithms used is adequate.
One thing that could be improved upon is the allocation of memory.
At runtime a lot of memory is allocated, reallocated and copied over.
Instead of keeping the digit arrays at the exact needed length, a function can be used to double the amount digits when needed.
Doubling the length of the array makes sure there are less memory allocations and the digit array is always large enough to contain all digits.\\
Another optimization would be to check in the multiplication algorithm for integers with value 0, which results in 0.
Multiplications with one of the operands being a single digit number is taken care of by the karatsuba algorithm.\\
One might consider using the shift methods for multiplying and dividing by the value of the base (or a power of the base), which is faster than the regular algorithms.
The downside of this approach is that it imposes a check when for each calculation which is only useful for very specific numbers.

\section{Program code}
%insert code here
\end{document}
